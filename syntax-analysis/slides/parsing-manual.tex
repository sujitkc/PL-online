\documentclass{beamer}
\usetheme{Antibes}
\usepackage{xcolor, colortbl}
\usepackage{algorithm}
\usepackage[noend]{algpseudocode}
\usepackage{textcomp}
\usepackage{listings}
\usepackage{hyperref}
\usepackage{alltt}
\usepackage{tikz}
\usepackage{framed}
\usepackage{marvosym}
\usepackage{wasysym}
\usepackage{marvosym}
\usepackage{crayola}
\usepackage{mathpartir}
\usepackage{tabularx}
\usepackage[belowskip=-15pt,aboveskip=0pt]{caption}
\usepackage[skins]{tcolorbox}
\usepackage{multicol}
\usetikzlibrary{positioning,shapes,arrows, backgrounds, fit, shadows, automata}
\usetikzlibrary{decorations.markings}
%\usepackage{wasysym}
%\usepackage{marvosym}
\setbeamertemplate{footline}[frame number]
%\usecolortheme{fly}
\usefonttheme{serif}

\title[Sujit]{Parsing}
\author{Sujit Kumar Chakrabarti}
\institute{IIITB}
\date{}
\begin{document}
\maketitle

\definecolor{lightblue}{rgb}{0.8,0.93,1.0} % color values Red, Green, Blue
\definecolor{darkblue}{rgb}{0,0,0.5} % color values Red, Green, Blue
\definecolor{Blue}{rgb}{0,0,1.0} % color values Red, Green, Blue
\definecolor{darkgreen}{rgb}{0,0.7,0.2} % color values Red, Green, Blue
\definecolor{Red}{rgb}{1,0,0} % color values Red, Green, Blue
\definecolor{Pink}{rgb}{0.7,0,0.2}
\definecolor{links}{HTML}{2A1B81}
\definecolor{mydarkgreen}{HTML}{126215}
\definecolor{mychoco}{HTML}{5C3317}

\newcommand{\myheader}[1]{
	{\color{purple}
		\begin{Large}
			\begin{center}
				{#1}
			\end{center}
		\end{Large}
	}
}
\newcommand{\myminorheader}[1]{
	{\color{purple}
		\begin{large}
			{#1}
		\end{large}
	}
}

\newcommand{\myprod}[0]{\hspace{0.5cm}$\rightarrow$\hspace{0.5cm}}
\newcommand{\mychoice}[0]{\hspace{0.75cm}$|$\hspace{0.25cm}}
\newcommand{\myderiv}[0]{\hspace{0.5cm}$\Rightarrow$\hspace{0.5cm}}


\tikzstyle{bb}=[%
      rectangle, draw=black, thick, fill=OliveGreen!30, drop shadow, align=center,
      text ragged, minimum height=2em, minimum width=2em, inner sep=6pt
]

\tikzstyle{inv}=[%
      rectangle, draw=none,  align=center,
      text ragged, minimum height=2em, minimum width=2em, align=center, inner sep=6pt
]

\tikzstyle{db}=[%
      ellipse, draw=black, thick, fill=pink, drop shadow, align=center,
      text ragged, minimum height=2em, inner sep=6pt
]

\tikzstyle{jn}=[%
      inner sep=0cm, outer sep=0cm
]

\tikzstyle{io}=[%
      trapezium, trapezium left angle=60, trapezium right angle=120, draw=black, thick, fill=brown, drop shadow,
      text ragged, minimum height=2em, minimum width=2em, inner sep=6pt, align=center
]

\tikzstyle{glio}=[%
      trapezium, trapezium left angle=60, trapezium right angle=120, draw=red, line width = 1mm, fill=brown, drop shadow,
      text ragged, minimum height=2em, minimum width=2em, inner sep=6pt
]
\tikzstyle{gl}=[%
      rectangle, draw=red, line width = 1mm, fill=lightblue, drop shadow,
      text ragged, minimum height=2em, minimum width=2em, inner sep=6pt
]

\tikzstyle{en}=[%
      rectangle, draw=black, thick, fill=none,
      text ragged, minimum height=2em, minimum width=2em, inner sep=6pt
]

\tikzstyle{sh}=[%
      rectangle, draw=gray, thick, fill=none, color = gray,
      text ragged, minimum height=2em, minimum width=2em, inner sep=6pt
]



\lstset{
	language = Caml,
	basicstyle = \ttfamily,
	stringstyle = \ttfamily,
	keywordstyle=\color{blue}\bfseries,
	identifierstyle=\color{mychoco},
	commentstyle=\color{darkgreen},
	frameround=tttt,
%	numbers=left,
	showstringspaces=false
}
\mathchardef\mhyphen="2D


% frame begin %%%%%%%%%%%%%%%%%%%%%%%%
\begin{frame}{Recursive Descent Parsing}

\pause
\begin{itemize}
	\item Appropriate for manual implementation
	\item Top-down parsing
	\item Starts with the root
	\item Prepares the parse tree in pre-order depth first sequence
	\item Finds the \textit{leftmost derivation} for a string
	\item LL($k$) Grammars
\end{itemize}
\end{frame}
% frame end %%%%%%%%%%%%%%%%%%%%%%%%

% frame begin %%%%%%%%%%%%%%%%%%%%%%%%
\begin{frame}{Leftmost and Rightmost Derivation}

\textbf{Grammar:} \\
\begin{tabular}{l @{} c @{} l}
$E$        & {\myprod}   & $E$ + $E$ $|$ ($E$) $|$ \textbf{id}
\end{tabular}

\textbf{Example:} \\
1 + (2 + 3) $\rightarrow$ $\textbf{id}_1$ + $\textbf{id}_2$ + $\textbf{Id}_3$

\vspace*{1cm}
\begin{tabular}{c @{} c}

\textbf{\color{blue}Leftmost Derivation} & \textbf{\color{blue}Rightmost Derivation} \\
\hline
\begin{minipage}{0.5\textwidth}
\begin{tabular}{l @{} c @{} l}
$E$     & \pause{\myderiv}   & $E$ + $E$                                             \\
        & \pause{\myderiv}   & $\textbf{id}_1$ + $E$                                 \\
        & \pause{\myderiv}   & $\textbf{id}_1$ + ($E$)                               \\
        & \pause{\myderiv}   & $\textbf{id}_1$ + ($E$ + $E$)                         \\
        & \pause{\myderiv}   & $\textbf{id}_1$ + ($\textbf{id}_2$ + $E$)             \\
        & \pause{\myderiv}   & $\textbf{id}_1$ + ($\textbf{id}_2$ + $\textbf{id}_3$)
\end{tabular}
\end{minipage}
&
\begin{minipage}{0.5\textwidth}
\begin{tabular}{l @{} c @{} l}
$E$     & \pause{\myderiv}   & $E$ + $E$                                             \\
        & \pause{\myderiv}   & $E$ + ($E$)                                           \\
        & \pause{\myderiv}   & $E$ + ($E$ + $E$)                                     \\
        & \pause{\myderiv}   & $E$ + ($E$ + $\textbf{id}_3$)                         \\
        & \pause{\myderiv}   & $E$ + ($\textbf{id}_2$ + $\textbf{id}_3$)                         \\
        & \pause{\myderiv}   & $\textbf{id}_1$ + ($\textbf{id}_2$ + $\textbf{id}_3$)
\end{tabular}
\end{minipage}

\end{tabular}

\end{frame}
% frame end %%%%%%%%%%%%%%%%%%%%%%%%

% frame begin %%%%%%%%%%%%%%%%%%%%%%%%
\begin{frame}{Leftmost and Rightmost Derivation}

\myheader{Formation of Parse Tree with Leftmost Derivation}
\begin{center}
\resizebox{!}{0.5\textheight}{%
\begin{tikzpicture}[auto,
    ->,
  %  shorten >=2pt,
    >=stealth,
  %  node distance=1cm,
    bb/.style={%
      rectangle, draw=black, very thick, fill=white,
      text ragged, minimum height=2em, inner sep=6pt, align=center
    },
    inv/.style={%
      rectangle, draw=none, fill=white,
      text ragged, minimum height=2em, inner sep=6pt, align=center
    },
    cinv/.style={%
      rectangle, draw=none, fill=white, text=white,
      text ragged, minimum height=2em, inner sep=6pt, align=center
    }
]
    \node[inv] (1)                        {$E$};
    \onslide<2->\node[inv] (2)  [below left = of 1]   {$E$};
    \onslide<2->\node[inv] (3)  [below = of 1]  {+};
    \onslide<2->\node[inv] (4)  [below right = of 1]  {$E$};

    \path (1) edge node {}  (2)
          (1) edge node {}  (3)
		  (1) edge node {}  (4)
	;

    \onslide<3->\node[inv] (5)  [below = of 2]  {$\textbf{id}_1$};
	\path (2) edge node {}  (5)
	;

    \onslide<4->\node[inv] (6)  [below left = of 4]  {(};
    \onslide<4->\node[inv] (7)  [below = of 4]  {$E$};
    \onslide<4->\node[inv] (8)  [below right = of 4]  {)};
	\path (4) edge node {}  (6)
		  (4) edge node {}  (7)
		  (4) edge node {}  (8)
	;

    \onslide<5->\node[inv] (9)  [below left = of 7]  {$E$};
    \onslide<5->\node[inv] (10)  [below = of 7]  {+};
    \onslide<5->\node[inv] (11)  [below right = of 7]  {$E$};
	\path (7) edge node {}  (9)
		  (7) edge node {}  (10)
		  (7) edge node {}  (11)
	;

    \onslide<6->\node[inv] (12)  [below = of 9]  {$\textbf{id}_2$};
	\path  (9) edge node {}  (12)
	;

    \onslide<7->\node[inv] (13)  [below = of 11]  {$\textbf{id}_3$};
	\path  (11) edge node {}  (13)
    ;

  \end{tikzpicture}

}
\end{center}

\end{frame}
% frame end %%%%%%%%%%%%%%%%%%%%%%%%

% frame begin %%%%%%%%%%%%%%%%%%%%%%%%
\begin{frame}{Leftmost and Rightmost Derivation}

\myheader{Formation of Parse Tree with Rightmost Derivation}
\begin{center}
\resizebox{!}{0.5\textheight}{%
\begin{tikzpicture}[auto,
    ->,
  %  shorten >=2pt,
    >=stealth,
  %  node distance=1cm,
    bb/.style={%
      rectangle, draw=black, very thick, fill=white,
      text ragged, minimum height=2em, inner sep=6pt, align=center
    },
    inv/.style={%
      rectangle, draw=none, fill=white,
      text ragged, minimum height=2em, inner sep=6pt, align=center
    },
    cinv/.style={%
      rectangle, draw=none, fill=white, text=white,
      text ragged, minimum height=2em, inner sep=6pt, align=center
    }
]
    \node[inv] (1)                        {$E$};
    \onslide<2->\node[inv] (2)  [below left = of 1]   {$E$};
    \onslide<2->\node[inv] (3)  [below = of 1]  {+};
    \onslide<2->\node[inv] (4)  [below right = of 1]  {$E$};

    \path (1) edge node {}  (2)
          (1) edge node {}  (3)
		  (1) edge node {}  (4)
	;

    \onslide<7->\node[inv] (5)  [below = of 2]  {$\textbf{id}_1$};
	\path (2) edge node {}  (5)
	;

    \onslide<3->\node[inv] (6)  [below left = of 4]  {(};
    \onslide<3->\node[inv] (7)  [below = of 4]  {$E$};
    \onslide<3->\node[inv] (8)  [below right = of 4]  {)};
	\path (4) edge node {}  (6)
		  (4) edge node {}  (7)
		  (4) edge node {}  (8)
	;

    \onslide<4->\node[inv] (9)  [below left = of 7]  {$E$};
    \onslide<4->\node[inv] (10)  [below = of 7]  {+};
    \onslide<4->\node[inv] (11)  [below right = of 7]  {$E$};
	\path (7) edge node {}  (9)
		  (7) edge node {}  (10)
		  (7) edge node {}  (11)
	;

    \onslide<6->\node[inv] (12)  [below = of 9]  {$\textbf{id}_2$};
	\path  (9) edge node {}  (12)
	;

    \onslide<5->\node[inv] (13)  [below = of 11]  {$\textbf{id}_3$};
	\path  (11) edge node {}  (13)
    ;

  \end{tikzpicture}

}
\end{center}

\end{frame}
% frame end %%%%%%%%%%%%%%%%%%%%%%%%

% frame begin %%%%%%%%%%%%%%%%%%%%%%%%
\begin{frame}{Recursive Descent Parsing}
{With backtracking -- Example}

\begin{tabular}{c @{\hspace{1cm}} c}
\myminorheader{Grammar:} & \myminorheader{Input string:} \\

\begin{minipage}{0.4\textwidth}

\begin{tabular}{l @{} c @{} l}
$S$ & {\myprod} & $c\ A\ d$ \\
$A$ & {\myprod} & $a\ b$    \\
$A$ & {\myprod} & $a$
\end{tabular}

\end{minipage}
&
\begin{minipage}{0.4\textwidth}
\begin{tabular}{l @{} c @{} l}

$w$ = \texttt{"cad"}
\end{tabular}
\end{minipage}
\end{tabular}

\end{frame}
% frame end %%%%%%%%%%%%%%%%%%%%%%%%

% frame begin %%%%%%%%%%%%%%%%%%%%%%%%
\begin{frame}{Recursive Descent Parsing}
{With backtracking -- Example}

\begin{center}
\begin{tabular}{c @{\hspace{1cm}} c}
\myminorheader{Grammar:} & \myminorheader{Input string:} \\

\begin{minipage}{0.4\textwidth}

\begin{tabular}{l @{} c @{} l}
$S$ & {\myprod} & $c\ A\ d$ \\
$A$ & {\myprod} & $a\ b$    \\
$A$ & {\myprod} & $a$
\end{tabular}

\end{minipage}
&
\begin{minipage}{0.4\textwidth}
\begin{tabular}{l @{} c @{} l}

$w$ = \texttt{"{\color{Red}\textbf{c}}ad"}
\end{tabular}
\end{minipage}
\end{tabular}
\end{center}

\myminorheader{Step 1:}

\begin{center}
\resizebox{0.3\textwidth}{!}{
\begin{tikzpicture}
\node[inv] (S) {${\color{Red}\textbf{S}}$};
\node[inv, below left = of S] (c) {${\color{Red}\textbf{c}}$};
\node[inv, below = of S] (A) {$A$};
\node[inv, below right = of S] (d) {$d$};

\draw[->, Blue](S) -- (c);
\draw[->, Blue](S) -- (A);
\draw[->, Blue](S) -- (d);

\end{tikzpicture}
}
\end{center}

\end{frame}
% frame end %%%%%%%%%%%%%%%%%%%%%%%%

% frame begin %%%%%%%%%%%%%%%%%%%%%%%%
\begin{frame}{Recursive Descent Parsing}
{With backtracking -- Example}

\begin{center}
\begin{tabular}{c @{\hspace{1cm}} c}
\myminorheader{Grammar:} & \myminorheader{Input string:} \\

\begin{minipage}{0.4\textwidth}

\begin{tabular}{l @{} c @{} l}
$S$ & {\myprod} & $c\ A\ d$ \\
$A$ & {\myprod} & $a\ b$    \\
$A$ & {\myprod} & $a$
\end{tabular}

\end{minipage}
&
\begin{minipage}{0.4\textwidth}
\begin{tabular}{l @{} c @{} l}

$w$ = \texttt{"c{\color{Red}\textbf{a}}d"}
\end{tabular}
\end{minipage}
\end{tabular}
\end{center}

%\vspace{1cm}
\myminorheader{Step 2:}
\begin{center}
\resizebox{0.3\textwidth}{!}{
\begin{tikzpicture}
\node[inv] (S) {${\color{Red}\textbf{S}}$};
\node[inv, below left = of S] (c) {$c$};
\node[inv, below = of S] (A) {${\color{Red}\textbf{A}}$};
\node[inv, below right = of S] (d) {$d$};
\node[inv, below left = of A] (a) {${\color{Red}\textbf{a}}$};
\node[inv, below right = of A] (b) {b};

\draw[->, Blue](S) -- (c);
\draw[->, Blue](S) -- (A);
\draw[->, Blue](S) -- (d);
\draw[->, Blue](A) -- (a);
\draw[->, Blue](A) -- (b);

\end{tikzpicture}
}
\end{center}


\end{frame}
% frame end %%%%%%%%%%%%%%%%%%%%%%%%

% frame begin %%%%%%%%%%%%%%%%%%%%%%%%
\begin{frame}{Recursive Descent Parsing}
{With backtracking -- Example}

\begin{center}
\begin{tabular}{c @{\hspace{1cm}} c}
\myminorheader{Grammar:} & \myminorheader{Input string:} \\

\begin{minipage}{0.4\textwidth}

\begin{tabular}{l @{} c @{} l}
$S$ & {\myprod} & $c\ A\ d$ \\
$A$ & {\myprod} & $a\ b$    \\
$A$ & {\myprod} & $a$
\end{tabular}

\end{minipage}
&
\begin{minipage}{0.4\textwidth}
\begin{tabular}{l @{} c @{} l}

$w$ = \texttt{"ca{\color{Red}\textbf{d}}"}
\end{tabular}
\end{minipage}
\end{tabular}
\end{center}

\myminorheader{Step 3:}
\begin{center}
\resizebox{0.5\textwidth}{!}{
\begin{tikzpicture}
\node[inv] (S) {${\color{Red}\textbf{S}}$};
\node[inv, below left = of S] (c) {$c$};
\node[inv, below = of S] (A) {${\color{Red}\textbf{A}}$};
\node[inv, below right = of S] (d) {$d$};
\node[inv, below left = of A] (a) {$a$};
\node[inv, below right = of A] (b) {${\color{Red}\textbf{b}}$};


\draw[->, Blue](S) -- (c);
\draw[->, Blue](S) -- (A);
\draw[->, Blue](S) -- (d);
\draw[->, Blue](A) -- (a);
\draw[->, Blue](A) -- (b);

\node[inv, right = of d] (failure) {${\color{Red}\textbf{Failure}}$};

\end{tikzpicture}
}

\end{center}

\end{frame}
% frame end %%%%%%%%%%%%%%%%%%%%%%%%

% frame begin %%%%%%%%%%%%%%%%%%%%%%%%
\begin{frame}{Recursive Descent Parsing}
{With backtracking -- Example}

\begin{center}
\begin{tabular}{c @{\hspace{1cm}} c}
\myminorheader{Grammar:} & \myminorheader{Input string:} \\

\begin{minipage}{0.4\textwidth}

\begin{tabular}{l @{} c @{} l}
$S$ & {\myprod} & $c\ A\ d$ \\
$A$ & {\myprod} & $a\ b$    \\
$A$ & {\myprod} & $a$
\end{tabular}

\end{minipage}
&
\begin{minipage}{0.4\textwidth}
\begin{tabular}{l @{} c @{} l}

$w$ = \texttt{"c{\color{Red}\textbf{a}}d"}
\end{tabular}
\end{minipage}
\end{tabular}
\end{center}

\myminorheader{Step 4:}
\begin{center}
\resizebox{0.5\textwidth}{!}{
\begin{tikzpicture}
\node[inv] (S) {${\color{Red}\textbf{S}}$};
\node[inv, below left = of S] (c) {$c$};
\node[inv, below = of S] (A) {${\color{Red}\textbf{A}}$};
\node[inv, below right = of S] (d) {$d$};
\node[inv, below = of A] (a) {${\color{Red}\textbf{a}}$};


\draw[->, Blue](S) -- (c);
\draw[->, Blue](S) -- (A);
\draw[->, Blue](S) -- (d);
\draw[->, Blue](A) -- (a);

\node[inv, right = of d] (backtracking) {${\color{Red}\textbf{Backtracking}}$};

\end{tikzpicture}
}

\end{center}

\end{frame}
% frame end %%%%%%%%%%%%%%%%%%%%%%%%

% frame begin %%%%%%%%%%%%%%%%%%%%%%%%
\begin{frame}{Recursive Descent Parsing}
{With backtracking -- Example}

\begin{center}
\begin{tabular}{c @{\hspace{1cm}} c}
\myminorheader{Grammar:} & \myminorheader{Input string:} \\

\begin{minipage}{0.4\textwidth}

\begin{tabular}{l @{} c @{} l}
$S$ & {\myprod} & $c\ A\ d$ \\
$A$ & {\myprod} & $a\ b$    \\
$A$ & {\myprod} & $a$
\end{tabular}

\end{minipage}
&
\begin{minipage}{0.4\textwidth}
\begin{tabular}{l @{} c @{} l}

$w$ = \texttt{"ca{\color{Red}\textbf{d}}"}
\end{tabular}
\end{minipage}
\end{tabular}
\end{center}

\myminorheader{Step 5:}
\begin{center}
\resizebox{0.5\textwidth}{!}{
\begin{tikzpicture}
\node[inv] (S) {${\color{Red}\textbf{S}}$};
\node[inv, below left = of S] (c) {$c$};
\node[inv, below = of S] (A) {$A$};
\node[inv, below right = of S] (d) {${\color{Red}\textbf{d}}$};
\node[inv, below = of A] (a) {$a$};


\draw[->, Blue](S) -- (c);
\draw[->, Blue](S) -- (A);
\draw[->, Blue](S) -- (d);
\draw[->, Blue](A) -- (a);

\node[inv, right = of d] (success) {${\color{Red}\textbf{Success}}$};

\end{tikzpicture}
}

\end{center}

\end{frame}
% frame end %%%%%%%%%%%%%%%%%%%%%%%%

% frame begin %%%%%%%%%%%%%%%%%%%%%%%%
\begin{frame}{Recursive Descent Parsing}
{With backtracking}
\begin{scriptsize}
\begin{algorithmic}[0]
\Procedure{$S$}{$pos$}
  \If {\Call {match}{$pos$,'c'}}
    \If {\Call {$A_1$}{$pos + 1$}}
      \If {match($pos + 3$, 'd')}
        \State \textbf{return} \textbf{true}
      \Else
        \State \textbf{return} \texttt{false}
      \EndIf
    \ElsIf {(\Call {$A_2$} {$pos + 1$})}
      \If{\Call {match} {($pos + 2$, 'd')}}
        \State \textbf{return} \texttt{true}
      \EndIf
    \EndIf
  \Else
    \State \textbf{return} \texttt{false}
  \EndIf
\EndProcedure
\end{algorithmic}
%\end{multicols}
\end{scriptsize}
\end{frame}
% frame end %%%%%%%%%%%%%%%%%%%%%%%%

% frame begin %%%%%%%%%%%%%%%%%%%%%%%%
\begin{frame}{Recursive Descent Parsing}
{With backtracking}
\begin{scriptsize}
\begin{algorithmic}[0]
\Procedure{$A_1$}{$pos$} 
  \If{\Call{match}{$pos$, 'a'} and \Call{match}{$pos + 1$, 'b'}}
    \State \textbf{return} \texttt{true}
  \Else
    \State \textbf{return} \texttt{false}
  \EndIf
\EndProcedure
\Statex
\Procedure{$A_2$}{$pos$} 
  \If{\Call{match}{pos, 'a'}}
    \State \textbf{return} \texttt{true}
  \Else
    \State \textbf{return} \texttt{false}
  \EndIf
\EndProcedure
\end{algorithmic}
\end{scriptsize}
\end{frame}
% frame end %%%%%%%%%%%%%%%%%%%%%%%%

% frame begin %%%%%%%%%%%%%%%%%%%%%%%%
\begin{frame}{Recursive Descent Parsing}
{With backtracking}

\begin{itemize}
\item Powerful algorithm
\item Backtracking
\item Back and forth movement of input pointer
\item May lead to inefficiency and complexity
\end{itemize}
\end{frame}
% frame end %%%%%%%%%%%%%%%%%%%%%%%%

% frame begin %%%%%%%%%%%%%%%%%%%%%%%%
\begin{frame}{Recursive Descent Parsing}
{Predictive Parsing -- Example}

\textbf{\color{purple}Grammar:} \\
\begin{tabular}{l @{} c @{} l}
$E$        & {\myprod}   & \textbf{num} $E'$                      \\
$E'$       & {\myprod}   & + $E$ $E'$ $|$ $\epsilon$
\end{tabular}

\vspace{0.5cm}
\textbf{\color{purple}Example:} \\
1 + 2 + 3 $\rightarrow$ $\textbf{num}_1$ + $\textbf{num}_2$ + $\textbf{num}_3$

\end{frame}
% frame end %%%%%%%%%%%%%%%%%%%%%%%%

% frame begin %%%%%%%%%%%%%%%%%%%%%%%%
\begin{frame}{Recursive Descent Parsing}
{Predictive Parsing -- Example}

\begin{tabular}{c@{\hspace{1cm}}c}
\begin{minipage}{0.4\textwidth}
\textbf{\color{purple}Grammar:} \\
\begin{tabular}{l @{} c @{} l}
$E$        & {\myprod}   & \textbf{num} $E'$                      \\
$E'$       & {\myprod}   & + $E$ $E'$ $|$ $\epsilon$
\end{tabular}

\vspace{0.5cm}
\textbf{\color{purple}Example:} \\
$\textbf{num}_1$ + $\textbf{num}_2$ + $\textbf{num}_3$
\end{minipage}
&
\begin{minipage}{0.5\textwidth}
\begin{framed}
\resizebox{!}{0.75\textheight}{
\begin{tikzpicture}
\node[inv](e1){$E$};

\pause

\node[inv, below left = of e1](n1){$n_1$};
\node[inv, below right = of e1](ed1){$E'$};

\draw[->, blue] (e1) -- (n1);
\draw[->, blue] (e1) -- (ed1);

\pause

\node[inv, below left = of ed1](p1){$+$};
\node[inv, below = of ed1](e2){$E$};
\node[inv, below right = of ed1, xshift = 1cm](ed2){$E'$};

\draw[->, blue] (ed1) -- (p1);
\draw[->, blue] (ed1) -- (e2);
\draw[->, blue] (ed1) -- (ed2);

\pause

\node[inv, below left = of e2](n2){$n_2$};
\node[inv, below right = of e2](ed3){$E'$};

\draw[->, blue] (e2) -- (n2);
\draw[->, blue] (e2) -- (ed3);

\pause

\node[inv, below left = of ed3](p2){$+$};
\node[inv, below = of ed3](e3){$E$};
\node[inv, below right = of ed3, xshift = 1cm](ed4){$E'$};

\draw[->, blue] (ed3) -- (p2);
\draw[->, blue] (ed3) -- (e3);
\draw[->, blue] (ed3) -- (ed4);

\pause

\node[inv, below left = of e3](n3){$n_3$};
\node[inv, below right = of e3](ed5){$E'$};

\draw[->, blue] (e3) -- (n3);
\draw[->, blue] (e3) -- (ed5);

\pause

\node[inv, below = of ed5](ep1){$\epsilon$};
\draw[->, blue] (ed5) -- (ep1);

\pause

\node[inv, below = of ed4](ep2){$\epsilon$};
\draw[->, blue] (ed4) -- (ep2);

\pause

\node[inv, below = of ed2](ep3){$\epsilon$};
\draw[->, blue] (ed2) -- (ep3);

\end{tikzpicture}
}
\end{framed}
\end{minipage}
\end{tabular}

\end{frame}
% frame end %%%%%%%%%%%%%%%%%%%%%%%%

% frame begin %%%%%%%%%%%%%%%%%%%%%%%%
\begin{frame}{Recursive Descent Parsing}
{Algorithm}

\textbf{\color{purple}Grammar:} \\
\begin{tabular}{l @{} c @{} l}
$E$        & {\myprod}   & \textbf{num} $E'$                      \\
$E'$       & {\myprod}   & + $E$ $E'$ $|$ $\epsilon$
\end{tabular}
\pause

\vspace{0.5cm}
\begin{tiny}

\begin{framed}
\begin{algorithmic}[0]
\Procedure{$E$}{}
	\State {\bf return} \Call{match}{${\bf num}$} \textbf{and} \Call{E'}{}
\EndProcedure
\Statex
\Procedure{$E'$}{}
	\State {\bf return} \Call{$E'_1$}{} \textbf{or} \Call{$E'_2$}{}
\EndProcedure
\Statex
\Procedure{$E'_1$}{}
	\State {\bf return} \Call{match}{+} \textbf{and} \Call{$E$}{} \textbf{and} \Call{$E'$}{}
\EndProcedure
\Statex
\Procedure{$E'_2$}{}
	\State {\bf return} \textbf{true}
\EndProcedure

\end{algorithmic}
\end{framed}
\end{tiny}

\end{frame}
% frame end %%%%%%%%%%%%%%%%%%%%%%%%

% frame begin %%%%%%%%%%%%%%%%%%%%%%%%
\begin{frame}{Recursive Descent Parsing}
{Algorithm -- Activity}

\textbf{\color{purple}Grammar:} \\
\begin{tabular}{l @{} c @{} l}
$expr$        & {\myprod}   & $\textbf{num}$ $|$ $term$                    \\
$term$        & {\myprod}   & $factor$ $|$ $factor$ + $term$               \\
$factor$      & {\myprod}   & $\textbf{num}$ $|$ $\textbf{num}$ * $factor$
\end{tabular}

\vfill

\end{frame}
% frame end %%%%%%%%%%%%%%%%%%%%%%%%

% frame begin %%%%%%%%%%%%%%%%%%%%%%%%
\begin{frame}{Recursive Descent Parsing}
{Left Recursion}
\textbf{\color{purple}Grammar:} \\
\begin{tabular}{l @{} c @{} l}
$E$        & {\myprod}   & $E$ + $T$ $|$ $\textbf{num}$
\end{tabular}
\pause
\vspace{0.5cm}
\begin{tiny}

\begin{framed}
\begin{algorithmic}[0]
\Procedure{$E$}{}
	\State {\bf return} \Call{$E$}{} \textbf{and} \Call{match}{+} \textbf{and} \Call{match}{${\bf num}$}
\EndProcedure
\Statex
\Procedure{$T$}{}
	\State {\bf return} \Call{$E'_1$}{} \textbf{or} \Call{$E'_2$}{}
\EndProcedure
\end{algorithmic}
\end{framed}
\end{tiny}

\end{frame}
% frame end %%%%%%%%%%%%%%%%%%%%%%%%


% frame begin %%%%%%%%%%%%%%%%%%%%%%%%
\begin{frame}{Recursive Descent Parsing}
{Left Recursion}

\begin{tabular}{c @{\hspace{0.5cm} $\Rightarrow$ \hspace{0.5cm}} c}
\begin{minipage}{0.45\textwidth}
\textbf{\color{purple}\underline{Grammar:}}
\begin{framed}
\begin{tabular}{l @{} c @{} l}
$E$        & {\myprod}   & $E$ + $T$ $|$ $\textbf{num}$
\end{tabular}
\end{framed}
\pause
\end{minipage}
&
\begin{minipage}{0.4\textwidth}
\textbf{\color{purple}\underline{Modified Grammar:}}

\begin{framed}
\begin{tabular}{l @{} c @{} l}
$E$        & {\myprod}   & \textbf{num} $E'$                      \\
$E'$       & {\myprod}   & + $E$ $E'$ $|$ $\epsilon$
\end{tabular}
\end{framed}
\end{minipage}
\end{tabular}

\pause
\begin{itemize}
\item Algorithm available for removing left recursion
\item Self-study
\end{itemize}
\end{frame}
% frame end %%%%%%%%%%%%%%%%%%%%%%%%

\end{document}
