\documentclass{beamer}
\usetheme{Antibes}
\usepackage{xcolor, colortbl}
\usepackage{algorithm}
\usepackage[noend]{algpseudocode}
\usepackage{textcomp}
\usepackage{listings}
\usepackage{hyperref}
\usepackage{alltt}
\usepackage{tikz}
\usepackage{framed}
\usepackage{marvosym}
\usepackage{wasysym}
\usepackage{marvosym}
\usepackage{crayola}
\usepackage{mathpartir}
\usepackage{tabularx}
\usepackage[belowskip=-15pt,aboveskip=0pt]{caption}
\usepackage[skins]{tcolorbox}
\usepackage{multicol}
\usetikzlibrary{positioning,shapes,arrows, backgrounds, fit, shadows}
\usetikzlibrary{decorations.markings}
%\usepackage{wasysym}
%\usepackage{marvosym}
\setbeamertemplate{footline}[frame number]
%\usecolortheme{fly}
\usefonttheme{serif}

\title[Sujit]{Programming Language Implementation -- Introduction}
\author{Sujit Kumar Chakrabarti}
\institute{IIITB}
\date{IIIT-B}

\begin{document}
\setbeamertemplate{navigation symbols}{}%remove navigation symbols

\maketitle

\definecolor{lightblue}{rgb}{0.8,0.93,1.0} % color values Red, Green, Blue
\definecolor{darkblue}{rgb}{0.4,0.3,1.0} % color values Red, Green, Blue
\definecolor{Blue}{rgb}{0,0,1.0} % color values Red, Green, Blue
\definecolor{darkgreen}{rgb}{0,0.7,0.2} % color values Red, Green, Blue
\definecolor{Red}{rgb}{1,0,0} % color values Red, Green, Blue
\definecolor{Pink}{rgb}{0.7,0,0.2}
\definecolor{links}{HTML}{2A1B81}
\definecolor{mydarkgreen}{HTML}{126215}
\newcommand{\highlight}[1]{{\color{Red}(#1)}}

\newcommand{\myheader}[1]{
	{\color{darkblue}
		\begin{Large}
			\begin{center}
				{#1}
			\end{center}
		\end{Large}
	}
}
\newcommand{\myminorheader}[1]{
	{\color{BrickRed}
		\begin{Large}
			{\fontfamily{\sfdefault}\selectfont\textbf{#1}}
		\end{Large}
	}
}

%\tikzstyle{input} = [coordinate]
%\tikzstyle{output} = [coordinate]


\tikzstyle{bb}=[%
      rectangle, draw=black, thick, fill=OliveGreen!30, drop shadow, align=center,
      text ragged, minimum height=2em, minimum width=2em, inner sep=6pt
]

\tikzstyle{inv}=[%
      rectangle, draw=none,  align=center,
      text ragged, minimum height=2em, minimum width=2em, inner sep=6pt
]

\tikzstyle{db}=[%
      ellipse, draw=black, thick, fill=pink, drop shadow, align=center,
      text ragged, minimum height=2em, inner sep=6pt
]

\tikzstyle{jn}=[%
      ellipse, draw=black, thick, fill=black
]

\tikzstyle{io}=[%
      trapezium, trapezium left angle=60, trapezium right angle=120, draw=black, thick, fill=brown, drop shadow,
      text ragged, minimum height=2em, minimum width=2em, inner sep=6pt, align=center
]

\tikzstyle{glio}=[%
      trapezium, trapezium left angle=60, trapezium right angle=120, draw=red, line width = 1mm, fill=brown, drop shadow,
      text ragged, minimum height=2em, minimum width=2em, inner sep=6pt
]
\tikzstyle{gl}=[%
      rectangle, draw=red, line width = 1mm, fill=lightblue, drop shadow,
      text ragged, minimum height=2em, minimum width=2em, inner sep=6pt
]

\tikzstyle{en}=[%
      rectangle, draw=black, thick, fill=none,
      text ragged, minimum height=2em, minimum width=2em, inner sep=6pt
]

\tikzstyle{sh}=[%
      rectangle, draw=gray, thick, fill=none, color = gray,
      text ragged, minimum height=2em, minimum width=2em, inner sep=6pt
]


\lstdefinestyle{javacode}{
	language = Java,
	basicstyle = \ttfamily\tiny,
	stringstyle = \ttfamily,
	keywordstyle=\color{Blue}\bfseries,
	identifierstyle=\color{Pink},
	commentstyle=\color{darkgreen},
	frameround=tttt,
%	numbers=left
	showstringspaces=false
}

\lstdefinestyle{camlcode}{
	language = Caml,
	basicstyle = \tiny\ttfamily,
	stringstyle = \color{red}\ttfamily,
	keywordstyle=\color{Blue}\bfseries,
	identifierstyle=\ttfamily,
	frameround=tttt,
	numbers=none,
	showstringspaces=false,
	escapeinside={(*@}{@*)}
}

\lstdefinestyle{outputcode}{
	language = bash,
	backgroundcolor = \color{black},
	basicstyle = \tiny\ttfamily\color{white},
	stringstyle = \color{red}\ttfamily,
	keywordstyle=\color{white}\bfseries,
	identifierstyle=\ttfamily,
	frameround=tttt,
	numbers=none,
	showstringspaces=false,
	escapeinside={(*@}{@*)}
}

\newtcolorbox{myframe}[2][]{%
  enhanced,colback=white,colframe=black,coltitle=black,
  sharp corners,boxrule=0.4pt,
  fonttitle=\itshape,
  attach boxed title to top left={yshift=-0.3\baselineskip-0.4pt,xshift=2mm},
  boxed title style={tile,size=minimal,left=0.5mm,right=0.5mm,
    colback=white,before upper=\strut},
  title=#2,#1
}

% frame begin %%%%%%%%%%%%%%%%%%%%%%%%
\begin{frame}{Recap}

So far, we have learned:
\begin{enumerate}
\item A set of programming language features (mostly related to types) that could help in classification of PLs
\item Functional programming
\item Logic programming
\end{enumerate}
\pause
Helpful in:
\begin{enumerate}
\item Learning a new PL
\item Choice of a new PL for programming
\item Programming Language Design and implementation (PLDI\footnote{inspired from a premier conference in the area of programming languages})
\end{enumerate}

\end{frame}
% frame end %%%%%%%%%%%%%%%%%%%%%%%%

% frame begin %%%%%%%%%%%%%%%%%%%%%%%%
\begin{frame}{Preview of the Rest of the Course}
{PLDI}

\begin{enumerate}
\item Knowledges + skills for implementing  programming languages
\item \textbf{Language processors.}
	\begin{itemize}
	\item Software systems embodying the concepts of a programming language
	\item Input: text in the implemented (input) language
	\item Output: Some action based on the input 
	\end{itemize}
\item \textbf{Examples.} Compilers, interpreters, editors, IDEs, browswers, RDBMS query processors, OS command processors etc.
\end{enumerate}

\end{frame}
% frame end %%%%%%%%%%%%%%%%%%%%%%%%

% frame begin %%%%%%%%%%%%%%%%%%%%%%%%
\begin{frame}{Language Processors}

\begin{center}
\begin{tikzpicture}
\node[inv](inp) {$I \in L$};
\node[bb, right=of inp](lp) {Language \\ Processor};
\node[inv, right=of lp](outp) {$O$};

\draw[->, thick, Maroon] (inp) -- (lp);
\draw[->, thick, Maroon] (lp) -- (outp);
\end{tikzpicture}
\end{center}
\begin{scriptsize}

\pause
\begin{itemize}
\item Input ($I$), Input text
\item Input Language ($L$), Implemented language
\item Output ($O$)
\end{itemize}

\pause
\myminorheader{Two Languages:}
\begin{itemize}
\item Input Language ($L$), Implemented language
\item Implementation Language
\pause
\item Examples:
\begin{center}
\begin{tabular}{|c|c|c|}
\hline
 & Input Language & Implementation Language \\
\hline
GCC & C & C \\
COQ & Gallina & OCaml \\
ESTEREL & ESTEREL & OCaml \\
\hline
\end{tabular}
\end{center}
\end{itemize}
\end{scriptsize}
\end{frame}
% frame end %%%%%%%%%%%%%%%%%%%%%%%%


% frame begin %%%%%%%%%%%%%%%%%%%%%%%%
\begin{frame}{Next Module}

\myheader{An Example of Language Processors -- Compilers}
\end{frame}
% frame end %%%%%%%%%%%%%%%%%%%%%%%%

\end{document}